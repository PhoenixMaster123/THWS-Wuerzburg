\documentclass{scrartcl}
\usepackage[ngerman]{babel}
\usepackage{amsmath}
\usepackage{amssymb}
\usepackage[utf8]{inputenc}
\usepackage[T1]{fontenc}

\title{Beispiele für mathematische Symbole und Formeln}

\begin{document}
\maketitle	
\section{Symbole}

\begin{itemize}
\item Griechisches Alphabet: $\alpha, \beta, \gamma, \dots$
\item Binäre Relationen: $\neq, \geq, \leq, \approx, \in, \notin, \dots$
\item Binäre Operatoren: $\vee, \wedge, \neg, \times, \cdot, \dots$
\item Pfeile: $\rightarrow, \rightarrowtail, \Rightarrow, \hookrightarrow, \dots$
\end{itemize}

\section{Abgesetzte Formeln}

\begin{equation}\label{eqn:einstein}
 E = mc^2
\end{equation}

\section{Einfache Formatierungen}

\begin{itemize}
\item Exponenten: Mit dem Zirkumflex: \texttt{a\^{}2} ergibt \( a^2 \)
\vspace{5mm}
\item Indizes: Mit dem Unterstrich: \texttt{a\_i} ergibt \( a_i \)
\vspace{5mm}
\item Brüche: \texttt{\textbackslash frac\{a\}\{b\}} ergibt \(\frac{a}{b}\)
\vspace{5mm}
\item Wurzeln: \texttt{\textbackslash sqrt\{a\}} ergibt \(\sqrt{a}\)
\vspace{5mm}
\item Summen: \texttt{\textbackslash sum\_\{i=1\}\^{}n} ergibt \(\sum_{i=1}^{n}\)
\vspace{5mm}
\item LaTeX optimiert automatisch die Darstellung% von Brüchen und Wurzeln.
\end{itemize}

\section{Größenanpassung bei Klammern}

\[
\left( \frac{a}{b} \right) 
\]

\section{Matrizen}

\begin{equation*}	
\begin{pmatrix}
1 & 2 \\
3 & 4
\end{pmatrix}
\end{equation*}

\section{Landau Notation}

\begin{itemize}
\item Landau-Notation mit dem "großen O": $\mathcal{O}$
\item Schreibeweise in LaTex: \texttt{\textbackslash mathcal\{O\}}
\vspace{5mm}
\item \textbf{Definition}: $f \in \mathcal{O}(g(n)):$
\item  $\exists\  c > 0, n_0 \in \mathbb{N}: \forall\ n > n_0: f(n) \leq c \cdot g(n)$
\vspace{5mm}
\end{itemize}
	
\end{document}
