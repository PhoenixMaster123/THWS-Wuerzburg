\documentclass[11pt,parskip]{scrartcl}
\usepackage[utf8]{inputenc}
\usepackage[ngerman]{babel} 
\usepackage{titling} 
%\usepackage[backend=biber, style=authoryear]{biblatex}
%\usepackage{csquotes}
\usepackage{xcolor}
\usepackage{enumitem}
\setlist[enumerate,1]{label=\Alph*)}
\usepackage{amssymb}
\usepackage{booktabs}
\usepackage{array}
\usepackage{amsmath}
\usepackage{listings}
\usepackage{lmodern}

\title{Bash and C Questions)}
\author{}
\date{}


\begin{document}

\maketitle

\section*{Bash (Shell Script)}

\subsection*{UNIT 5}

\subsection*{Question 1}
\textbf{Which command is used to find if a program with the same name as a script exists?} \\
    Outputs the full path of the command specified as an argument. \\
   \textbf{Example:} which ls $\rightarrow$ /bin/ls
   
\subsection*{Question 2}
\textbf{Which command is used to make a script executable?} \\
chmod u+x scriptname.sh

\subsection*{Question 3}
\textbf{How can you execute a script named "example.sh" from the current directory?} \\
./example.sh

\subsection*{Question 4}
\textbf{What is the purpose of the She-Bang line in a shell script?} \\
It defines the program to execute the script.

\subsection*{Question 5}
\textbf{What is Bash?} \\
Turing-complete programming language

\subsection*{Question 6}
\textbf{Which syntax is correct fro defining a variable in a Bash script?} \\
VARIABLE=5 \\
\textbf{How to access the value of a variable?} \\
\$VARIABLE

\subsection*{Question 7} 
\textbf{What does the command read -s do in Bash scripting?} \\
Reads input without showing displaying it. \\
\textbf{What does the command read -t do in Bash scripting?} \\
Set a timeout on read to limit the time taken to input text: \\
\textbf{What does the command read -n do in Bash scripting?} \\
N only read given number of characters

\subsection*{Question 8}
\textbf{\$\#} contains the number of arguments \\
\textbf{\$}$*$ contains all arguments as one string \\
\textbf{\$}$@$ contains all arguments as string array

\section*{UNIT 6}
\subsection*{INFO:}
$egrep [options] [search\_term] [location]$ \\

Regular Expressions: \\
Matching every character $\rightarrow$ . \\

\begin{table}[h!]
\centering
\begin{tabular}{@{}ll@{}}
\toprule
\textbf{Regex Class} & \textbf{Description} \\ \midrule
\texttt{\textbackslash d}   & Matches any digit (0-9) \\
\texttt{\textbackslash w}   & Matches any digit, letter, or underscore \\
\texttt{\textbackslash s}   & Matches any whitespace character (space, tab, newline) \\
\texttt{\textbackslash D}   & Matches any character that is \textbf{not} a digit \\
\texttt{\textbackslash W}   & Matches any character that is \textbf{not} a digit, letter, or underscore \\
\texttt{\textbackslash S}   & Matches any character that is \textbf{not} whitespace \\ \bottomrule
\end{tabular}
\caption{Regex Character Classes and Their Descriptions}
\end{table}

\begin{table}[h!]
\centering
\begin{tabular}{@{}ll@{}}
\toprule
\textbf{Regex Quantifier} & \textbf{Description} \\ \midrule
\texttt{?}   & Matches \textbf{zero or one} occurrence of the preceding element \\
\texttt{*}   & Matches \textbf{zero or many} occurrences of the preceding element \\
\texttt{+}   & Matches \textbf{one or many} occurrences of the preceding element \\ \bottomrule
\end{tabular}
\caption{Regex Quantifiers and Their Descriptions}
\end{table}


\begin{table}[h!]
\centering
\begin{tabular}{@{}ll@{}}
\toprule
\textbf{Regex Anchor} & \textbf{Description} \\ \midrule
\texttt{\^}   & Matches the beginning of a line \\
\texttt{\$}   & Matches the end of a line \\
\texttt{\textbackslash n} & Matches a new line character \\ \bottomrule
\end{tabular}
\caption{Regex Line Anchors and Their Descriptions}
\end{table}


\textbf{REMOVE STRING:} \\

\begin{table}[ht]
\centering
\begin{tabular}{|c|c|}
\hline
\textbf{Operation} & \textbf{Syntax and Description} \\
\hline
\textbf{Cut at the beginning} & \texttt{\$\{VAR1\#Hello\}} \quad $\rightarrow$ \text{Cut at the beginning} \\
\hline
\textbf{Cut at the end} & \texttt{\$\{VAR1\%Hello\}} \quad $\rightarrow$ \text{Cut at the end} \\
\hline
\end{tabular}
\caption{String Removal Operations}
\end{table}
\textbf{Remove Substrings by Index and Length} \\

VAR1="Helloworld" \\

\begin{table}[ht]
\centering
\begin{tabular}{|c|c|}
\hline
\textbf{Operation} & \textbf{Result} \\
\hline
\texttt{\${VAR1:2}} & \text{lloworld} \\
\hline
\texttt{\${VAR1:0:1}} & \text{H} \\
\hline
\texttt{\${VAR1:1:3}} & \text{ell} \\
\hline
\end{tabular}
\caption{Array Manipulation Examples}
\end{table}

\textbf{If the substring is not part of the given string, the original value is returned.}

\textbf{Arrays:} \\

\begin{table}[ht]
\centering
\begin{tabular}{|c|c|}
\hline
\textbf{Operation} & \textbf{Syntax} \\
\hline
Length of an array & \texttt{\${\#VAR1[*]}} \\
\hline
Access all elements & \texttt{\${VAR1[*]}} \\
\hline
Copy an array & \texttt{VAR2=(\${VAR1[*]})} \\
\hline
\end{tabular}
\caption{Array Operations in Shell}
\end{table}

\subsection*{Question 1}
\textbf{Which of the following represents the use of the ? quantifier in regular expression?} \\
Matching zero or once occurrence.

\subsection*{Question 2}
\textbf{How do you find the string length stored in a Bash variable?} \\
\$\{\#Var\}

\subsection*{Question 3}
\textbf{How can you define an array in Bash?}
%VAR1=(\textquote{one} \textquote{two} \textquote{three})
VAR1=("'one"' "'two"' "'three"')

\section*{UNIT 7}

\subsection*{INFO:}

\begin{lstlisting}[language=bash, basicstyle=\ttfamily, frame=single]
#!/bin/bash

# Define a variable
echo "Enter a number (1-3):"
read number

# Switch-case to evaluate the input
case $number in
    1)
        echo "You entered One."
        ;;
    2)
        echo "You entered Two."
        ;;
    3)
        echo "You entered Three."
        ;;
    *)
        echo "Invalid input."
        ;;
esac
\end{lstlisting}


\begin{lstlisting}[language=bash]
for ((i=0; i<=9; i++))
do
  echo "Hello World " $i
done

for i in 0..9
do
  echo "Hello World " $i
done

for file in *.txt
do
  echo "Filename: " $file
done

for name in $(find . -name *.jpg)
do
  echo "JPG Bilddatei: " $name
done
\end{lstlisting}


\subsection*{Question 1}
\textbf{How do you define constant in Bash?} \\
readonly VARIABLE=value

\subsection*{Question 2}
\textbf{Which command makes a variable visible to subscript?} \\
export VAR

\subsection*{Question 3}
\textbf{How can you perform arithmetic operations in Bash?} \\
Using expr

\subsection*{Question 4}
\textbf{What is the behaviour of the command test \$VAR=value in Bash} \\
It checks if VAR is equal to value

\subsection*{Question 5}
\textbf{What is the correct syntax for a Bash for-loop over a range of numbers?} \\

\begin{lstlisting}[language=bash]
for ((i=0; i<=9; i++))
do
  echo "Hello World " $i
done
\end{lstlisting}

\subsection*{Question 6}
\textbf{How can you define and use a function in Bash?} \\
\begin{lstlisting}[language=bash]
foo() { echo "Hello"; }
\end{lstlisting}

\section*{UNIT 8}

\subsection*{Question 1}
\textbf{Which command is primarily used to retrieve data from a specified URL?} \\
\textbf{Answer:} curl

\subsection*{Question 2}
\textbf{If you need to replace all occurrences of a character in a file with another character, which command would you use?} \\
\textbf{Answer:} tr

\subsection*{Question 3}
\textbf{Which command is used to extract specific columns or fields from a file?} \\
\textbf{Answer:} cut

\subsection*{Question 4} 
\textbf{To arrange lines of text in a file alphabetically or numerically, you would use the ...... command.} \\
\textbf{Answer:} sort

\subsection*{Question 5}
\textbf{If you have two files with a common key field, and you want to merge them based on that key, which command would you use?} \\
\textbf{Answer:} join

\subsection*{Question 6} 
\textbf{Which command is specifically designed to format data into neat columns for better readability?} \\
\textbf{Answer:} column

\subsection*{Question 7}
\textbf{Which of the following is the syntax for a GET request using curl?} \\
\textbf{Answer:} curl [options] URL

\subsection*{Question 8} 
\textbf{Which of the following options is used to delete specified characters in the 'tr' command?} \\
\textbf{Answer:} -d

\subsection*{Question 9} 
\textbf{What does the 'cut -d ',' -f1,2' command do?} \\
\textbf{Answer:} Extracts the first and second field based on comma delimiter

\subsection*{Question 10} 
\textbf{How can you sort lines of a file numerically in reverse order using the 'sort' command?} \\
\textbf{Answer:} sort -r -n

\subsection*{Question 11}
\textbf{What is the primary function of the sed command?} \\
\textbf{Answer:} Editing text files in a non-interactive way.

\subsection*{Question 12}
\textbf{What does the s command in sed typically represent?} \\
\textbf{Answer:} Substitute

\subsection*{Question 13}
\textbf{Which of the following sed commands would substitute all occurrences of "foo" with "bar" in a file}
\begin{lstlisting}[language=bash]
sed 's/foo/bar/g' file
\end{lstlisting}

\subsection*{Question 14} 
\textbf{How can you print the lines of a file that match a specific pattern using sed?}
\textbf{Answer:} Use the -n option with the p flag to print only matching lines.

\subsection*{Question 15} 
\textbf{What is the primary purpose of the AWK command?} \\
\textbf{Answer:} Scanning and processing text data

\subsection*{Question 16}
\textbf{How does AWK typically process input data?} \\
\textbf{Answer:} By lines

\subsection*{Question 17}
\textbf{What is the default field separator in AWK?} \\
\textbf{Answer:} Space

\subsection*{Question 18}
\textbf{Which of the following is used to access the first field in a record in AWK?} \\
\textbf{Answer:} \$1

\subsection*{Question 19}
\textbf{How do you print the entire record in AWK?} \\
\textbf{Answer:} \$0

\subsection*{Question 20}
\textbf{Which of the following is used to define a variable in AWK?} \\
\textbf{Answer:} name = value

\subsection*{Question 21}
\textbf{What is the purpose of the BEGIN block in an AWK script?} \\
\textbf{Answer:} To execute code after processing all input lines

\subsection*{Question 22} 
\textbf{Which of the following is used to define a function in AWK?} \\
\begin{lstlisting}[language=bash]
function name(arguments) {
  ...
}
\end{lstlisting}

\subsection*{Question 23}
\textbf{Which operator is used for string concatenation in AWK?} \\
\textbf{Answer:} \&

\subsection*{Question 24}
\textbf{How can you compare two strings for equality in AWK?} \\
\textbf{Answer} ==

\subsection*{Question 25}
\textbf{What is the primary purpose of the jq command?}\\
\textbf{Answer:} Manipulating and filtering JSON data

\subsection*{Question 26}
\textbf{How does jq typically process JSON data?} \\
\textbf{Answer:} As a stream of tokens

\subsection*{Question 27}
\textbf{Which of the following is used to access a specific key within a JSON object in jq?} \\
Answer: .key

\subsection*{Question 28}
How do you access an array element at index 0 in jq? \\
Answer: a) .[0]

\subsection*{Question 29}
Which operator is used to filter elements from an array in jq? \\
Answer: ?

\subsection*{Question 30}
What does the following jq expression do: . \\
Answer: a) Accesses the current input value

\subsection*{Question 31}
How do you add a new key-value pair to a JSON object in jq? \\
Answer: a) .key = value

\subsection*{Question 31}
Which of the following is used to create a new JSON object in jq? \\
Answer: a) {}

\subsection*{Question 32}

What does the following jq expression do: | .[] \\
Answer: b) Iterates over each element of an array

\subsection*{Question 33}
How do you convert a JSON object to a string in jq? \\
\begin{lstlisting}[language=bash]
.to_string()
\end{lstlisting}

\section*{UNIT 11}

\subsection*{Question 1}
\textbf{Who is considered the main developer of the C programming language?} \\
\textbf{Answer: } Dennis Ritchie

\subsection*{Question 2}
\textbf{How C Language is compiled?} \\
\textbf{Answer:} In binary code

\subsection*{Question 3}
\textbf{How many bytes is the data type char?} \\
\textbf{Answer:} 1 Byte

\subsection*{Question 4} 
\textbf{What is the correct way to read input from the user using the scanf() function for an integer value?} \\
\begin{lstlisting}[language=bash]
scanf(\& age, "%d");
\end{lstlisting}
\textbf{Note:} scanf can only read data, it can not write data


\subsection*{Question 5} 
\textbf{What the data type void represents?} \\
\textbf{Answer:} empty or null

\subsection*{Question 6}
\textbf{Which operator is used to determine the memory address of a variable in C?} \\
\textbf{Answer:} \&

\section*{UNIT 12}

\subsection*{Question 1}
\textbf{Which storage class retains its value between function calls?} \\
\textbf{Answer:} static

\subsection*{Question 2}
\textbf{What is the default storage class for variables in C?} \\
\textbf{Answer:} auto

\subsection*{Question 3}
\textbf{Variables declared with which storage class are intended to be stored in CPU registers for faster access?} \\
\textbf{Answer:} register

\subsection*{Question 4}
\textbf{Which storage class is used for variables that can change unexpectedly due to external factors or hardware interrupts?} \\
\textbf{Answer:} volatile

\subsection*{Question 5} 
\textbf{Which of the following is NOT a primitive data type in C?} \\
\textbf{Answer:} string

\subsection*{Question 6}
\textbf{What is the main function of the C preprocessor?} \\
\textbf{A symbolic constant that is replaced by its value during preprocessing}

\subsection*{Question 7}
\textbf{What is a macro in the context of C preprocessor?} \\
\textbf{Answer:} A symbolic constant that is replaced by its value during preprocessing

\subsection*{Question 8}
\textbf{How does the preprocessor handle macro replacements?} \\
\textbf{Answer:} Replaces the macro name with its value in all occurrences

\subsection*{Question 9}
\textbf{What happens when a static variable is declared inside a function?} \\
\textbf{Answer:} Its value is preserved across multiple calls to the function.


\subsection*{Question 10} 
\textbf{What happens if you write beyond the bounds of an array when reading from a file?} \\
\textbf{It can lead to a segmentation fault.}

\subsection*{Question 11}
\textbf{What is the purpose of the null-terminator (\texttt{\textbackslash0}) in a character array?} \\
\textbf{Answer:} To indicate the end of the string

\subsection*{Question 12} 
\textbf{How do you determine the length of a character array in C?} \\
\textbf{Answer} By counting the number of characters until the null-terminator (\texttt{\textbackslash0})

\subsection*{Question 13}
\textbf{Which function is generally preferred for reading an entire line of input from the user in C?} \\
\textbf{fgets}

\subsection*{Question 14} 
\textbf{What is the purpose of the typedef keyword when used with a struct?} \\
\textbf{Answer:} To create a new data type with a custom name

\subsection*{Question 15} 
\textbf{How can you access the members of a struct in C?} \\
\textbf{Answer:} Using the . (dot) operator

\subsection*{Question 16} 
\textbf{Which of the following statements about structs in C is TRUE?} \\

a) Structs have significant runtime overhead compared to classes in Java. \\
b) Structs provide encapsulation and data hiding like classes in Java. \\
c) Structs are simple data structures with no methods/overhead. \\
d) Structs cannot be used to group variables of different data types. \\

\textbf{Answer:} c) Structs are simple data structures with no methods.

\subsection*{Question 17}
\textbf{When you assign one struct variable to another in C, what happens to the data?} \\

a) A reference to the original struct is created. \\
b) A pointer to the original struct is assigned. \\
c) The data from the original struct is copied to the new struct. \\
d) The two structs share the same memory location. \\

\textbf{Answer:} c) The data from the original struct is copied to the new struct.

\section*{UNIT 13}

\subsection*{Question 1}
\textbf{What does a pointer variable in C store?} \\
\textbf{Answer:} The memory address of a variable

\subsection*{Question 2}
\textbf{What does adding 1 to a pointer in C do?} \\
\textbf{Answer:} It moves the pointer to the next memory location based on the data type.

\subsection*{Question 3}
\textbf{How are arrays and pointers related in C?} \\
\textbf{Answer:} The name of an array is a pointer to its first element

\subsection*{Question 4}
\textbf{What is the operator used to get the value stored at the address pointed to by a pointer?} \\
\textbf{Answer:} star

\subsection*{Question 5}
\textbf{What is the operator used to get the address of a variable?} \\
\textbf{Answer:} \&

\subsection*{Question 6}
\textbf{What is a function pointer in C?} \\
\textbf{Answer:} A pointer that stores the address of a function

\subsection*{Question 7}
\textbf{What is a common risk of using pointers in C?} \\
\textbf{Answer:} Accessing or modifying unintended memory areas

\section*{UNIT 15}

\subsection*{Question 1} 
\textbf{What is a thread in the context of a computer program?} \\
\textbf{Answer:} A sequential flow of execution within a process

\subsection*{Question 2} 
\textbf{Which function in C is used to create a new thread?} \\
\textbf{Answer:} pthread\_create

\subsection*{Question 3}
\textbf{Which function in C is used to wait until a thread is finished?} \\
\textbf{Answer:} pthread\_join

\subsection*{Question 4}
\textbf{Which function in C is used to kill a thread?} \\
\textbf{Answer:} pthread\_exit

\subsection*{Question 5} 
\textbf{What is the "Lost Update" problem in concurrent programming?} \\
\textbf{Answer:} When a thread's changes are overwritten by another thread \\
\textbf{Answer2:} Concurrent access and modification of shared data without synchronization.

\subsection*{Question 6}
\textbf{What is a critical section in concurrent programming?} \\
\textbf{Answer:} A piece of code that can only be executed by one thread at a time

\subsection*{Question 7}
\textbf{What is the purpose of using semaphores in concurrent programming?} \\
\textbf{Answer:} To prevent race conditions and ensure thread safety

\subsection*{Question 8}
\textbf{What is the difference between a binary semaphore and a counting semaphore?} \\
\textbf{Answer:} A binary semaphore can only have a value of 0 or 1, while a counting semaphore can have any non-negative integer value.

\subsection*{Question 9}
\textbf{What does the sem\_wait() function do in the provided code snippet?} \\
\textbf{Answer:} Decrements the semaphore value

\subsection*{Question 10} 
\textbf{What does the sem\_post() function do in the provided code snippet?} \\
\textbf{Answer:} increments the semaphore value

\subsection*{Question 11}
\textbf{What is a deadlock in the context of concurrent programming?} \\
\textbf{A situation where two or more threads are blocked indefinitely, waiting for resources held by each other.}

\subsection*{Question 12}
\textbf{What is another common name for a binary semaphore?} \\
\textbf{Answer:} Mutex

\subsection*{Question 13}
\textbf{What does the second parameter in the \texttt{sem\_init} function specify?} \\
\textbf{Whether the semaphore is shared between threads or processes}

\subsection*{Question 14}
\textbf{What is a major difference between processes and threads in C programming?} \\
\textbf{Answer:} Threads share the same memory space, while processes do not.

\subsection*{Question 15}
\textbf{How can deadlocks be avoided when using multiple semaphores?} \\
\textbf{Answer:} By ensuring all threads acquire semaphores in the same order

\section*{UNIT 17}

\subsection*{Question 1}
\textbf{Which function is commonly used to open a file in C?} \\
\textbf{Answer:} fopen()

\subsection*{Question 2}
\textbf{Which function is used to close a file that was previously opened using fopen()?} \\
\textbf{Answer:} fclose()

\subsection*{Question 3}
\textbf{Which function is used to read data from a file in C?} \\
\textbf{Answer:} fread()

\subsection*{Question 4}
\textbf{Which function is used to write data to a file in C?} \\
\textbf{Answer:} fwrite()

\subsection*{Question 5}
\textbf{Which file opening mode is used for writing to a file, creating it if it doesn't exist, and truncating it if it does?} \\
\textbf{Answer:} w

\subsection*{Question 6}
\textbf{What does the SEEK\_SET parameter in the fseek() function represent?} \\
\textbf{The beginning of the file}

\subsection*{Question 7}
\textbf{Which function is used to write a single character to a file in C?} \\
\textbf{Answer:} fputc

\subsection*{Question 8}
\textbf{What does the fgets() function do?} \\
\textbf{Answer:} Reads a line of text from a file

\subsection*{Question 9}
\textbf{Why is it important to call fclose() on a file that has been opened?} \\
\textbf{Answer:} \\
a) To prevent memory leaks
b) To ensure that all data is written to the disk
c) To free up system resources

\subsection*{Question 10}
\textbf{Answer:} a+

\end{document}