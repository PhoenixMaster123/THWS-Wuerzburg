\documentclass{article}
\usepackage{amsmath}
\usepackage{graphicx}
\usepackage{enumitem}
\usepackage{fancyhdr}
\usepackage{hyperref}
\usepackage{listings}
\usepackage{graphicx}
\usepackage{subcaption}

\pagestyle{fancy}
\fancyhf{}
\fancyhead[L]{Theory of OS}
\fancyhead[R]{Kristian Popov}

\begin{document}
\title{Theory of Operating System Preparation}
\author{Kristian Popov}
\date{}
\maketitle

\section*{UNIT 26}

\subsubsection*{Question 1} 
\textbf{What is a process?} \\
\textbf{Answer:} A program currently being executed.

\subsubsection*{Question 2}
\textbf{What happens when the same program is started twice?} \\
\textbf{Answer:} Two processes are created, but the code is not duplicated in memory.

\subsubsection*{Question 3}
\textbf{What does the process model contain?} \\
\textbf{Answer:} The main memory content, program counter, and all CPU registers. \\
\textbf{NOTE:} Each process has its \textbf{virtual CPU}

\subsubsection*{Question 4}
\textbf{What is the primary reason for inefficiency in strict sequential processing?} \\
\textbf{Answer:} Waiting times for I/O devices

\subsubsection*{Question 5}
\textbf{What is spooling?} \\
\textbf{Answer:} It allowed for overlapped execution of CPU and I/O operations.

\subsubsection*{Question 6}
\textbf{What is the key characteristic of multi-programming or multi-tasking?} \\
\textbf{Answer:} Multiple programs sharing a single CPU.

\subsubsection*{Question 7}
\textbf{In which mode do core operating system functions execute?} \\
\textbf{Answer:} Kernel Mode

\subsubsection*{Question 8}
\textbf{What kind of access does the kernel mode have?} \\
\textbf{Answer:} Full access to hardware and system resources.  \\
\textbf{NOTE:} All CPU processes can be executed including direct access to the memory

\subsubsection*{Question 9}
\textbf{Which of the following statements is true about user mode?} \\
\textbf{a)} All processes are strictly separated from each other. \\
\textbf{b)} One process can't access the memory of another process \\
\textbf{c)} Processes in user mode cannot modify interrupt handling \\
\textbf{Answer:} All of them

\subsubsection*{Question 10}
\textbf{How is called the time between interrupts?} \\
\textbf{Answer:} Quantum

\subsubsection*{Question 11}
\textbf{What is the primary consequence of a small quantum size?} \\
\textbf{Answer:} More frequent context switches

\subsubsection*{Question 12}
\textbf{Which component of the operating system is responsible for selecting the next process to execute?} \\
\textbf{Answer:} The scheduler

\subsubsection*{Question 13}
\textbf{In the context of user input, why is a small quantum beneficial?} \\
\textbf{Answer:} It ensures that all processes waiting for input can handle new data quickly.

\subsubsection*{Question 14}
\textbf{What is the primary effect of a large quantum size on process execution?} \\
\textbf{Answer:} Decreased frequency of context switches

\subsubsection*{Question 15}
\textbf{What is a potential consequence of a large quantum size for user interaction?} \\
\textbf{Answer:} Slower response times to user input.

\subsubsection*{Question 16}
\textbf{What is a Process Control Block (PCB)?} \\
\textbf{Answer:} A data structure that contains information about a running process.

\subsubsection*{Question 17}
\textbf{Where is the PCB for a process typically stored?} \\
\textbf{Answer:} Kernel space \\
\textbf{NOTE:} \textbf{Kernel space} is separated from the user processes. \\
\textbf{NOTE:} All data of the process and etc. are located int the \textbf{user space} \\
\textbf{NOTE:} The \textbf{metadata} about processes is stored in \textbf{kernel space}

\subsubsection*{Question 18}
\textbf{In which state does a process begin its lifecycle?} \\
\textbf{Answer:} Creating \\
\textbf{NOTE:} The system allocates the necessary resources.

\subsubsection*{Question 19}
\textbf{What happens to a process in the Ready state?} \\
\textbf{Answer:} It is ready to execute but waiting for CPU time.

\subsubsection*{Question 20}
\textbf{When does a process transition to the Blocked state?} \\
\textbf{Answer:} When it is waiting for I/O or an external event.

\subsubsection*{Question 21}
\textbf{What is the significance of the Terminated state?} \\
\textbf{Answer:} It marks the end of a process's lifecycle and resource reclamation.
 
\subsubsection*{Question 22}
\textbf{What is the primary function of the "fork" system call in Linux?} \\
\textbf{Answer:} To create a new child process that is a near-exact copy of the parent process. 
(parent-child relation)

\subsubsection*{Question 23}
\textbf{Why is the parent-child relationship between processes important?} \\
\textbf{Answer:} It allows the child process to inherit certain crucial attributes from the parent

\subsubsection*{Question 24}
\textbf{What is the purpose of the "execve" system call?} \\
\textbf{Answer:} Is used to start an entirely newprogram

\subsubsection*{Question 25}
\textbf{When a running process encounters a Timer interrupt, what state does it typically transition to?} \\
\textbf{Answer:} Ready state

\subsubsection*{Question 26}
\textbf{What is the primary role of Timer interrupts in Linux process management?} \\
\textbf{Answer:} To enforce time-sharing among processes by interrupting running processes.

\subsubsection*{Question 27}
\textbf{What happens to a running process when it initiates an I/O system call?} \\
\textbf{Answer:} It transitions to the Blocked state.

\subsubsection*{Question 28}
\textbf{What is the primary function of the scheduler in Linux?} \\
\textbf{Answer:} To select the next process to run from the pool of Ready processes.

\subsubsection*{Question 29}
\textbf{Who is responsible for handling interrupts in the operating system?} \\
\textbf{Answer:} The operating system kernel.

\subsubsection*{Question 30}
\textbf{After an interrupt signals the arrival of necessary data, what does the operating system do?} \\
\textbf{Answer:} It checks if all required data has been received and then transitions the process to the Ready state.

\subsubsection*{Question 31}
\textbf{What are the three primary reasons for a process to terminate?} \\
\textbf{Answer:} Regular termination, errors, and termination by the operating system.

\subsubsection*{Question 32}
\textbf{What is a "segmentation fault" and how does it affect a process?} \\
\textbf{Answer:} A segmentation fault is an error that occurs when a process attempts to access memory that it is not authorized to use.

\subsubsection*{Question 33}
\textbf{What is the role of the Dispatcher in process management?} \\
\textbf{Answer:} Gives control to the next process \\
\textbf{NOTE:} The dispatcher saves the state of the current process and loads the last state of the next process

\subsubsection*{Question 34}
\textbf{What are the key steps involved in a process switch?} \\
\textbf{Answer:} \\
1) Halted the current process and storing the state \\
2) Selecting the next process using some strategy \\
3) Loading the state of the next process (Current CPU cache gets invalid) \\
4) Executing the next process

\subsubsection*{Question 35}
\textbf{What is the trade-off involved in process scheduling?} \\
\textbf{Answer:} Balancing high system responsiveness with efficient CPU utilization and minimizing overhead.

\subsubsection*{Question 36}
\textbf{What distinguishes a CPU-bound process from an I/O-bound process?} \\
\textbf{Answer:} \\ 
CPU-bound $\rightarrow$ needs CPU much and needs less IO \\
IO-bound $\rightarrow$ do a lot of IO and don't need the CPU much \\
\textbf{NOTE:} IO-bound process is more \textbf{important}.

\subsubsection*{Question 37}
\textbf{What is a characteristic of non-preemptive scheduling strategies?} \\
\textbf{Answer:} The process releases the CPU voluntarily \\
\textbf{NOTE:} The process changes into the state Blocked.

\subsubsection*{Question 38}
\textbf{What is the key mechanism that triggers preemptive scheduling in an operating system?} \\
\textbf{Answer:} Timer interrupt (starts the scheduler) \\
\textbf{NOTE:} the process is interrupted by the operating system

\subsubsection*{Question 39}
\textbf{What is the primary goal of Batch Processing scheduling?} \\
\textbf{Answer:} patient users, long quantum periods \\

\subsubsection*{Question 40}
\textbf{What is the primary goal of Desktop scheduling?} \\
\textbf{Answer:} high responsiveness, short quantum periods \\

\subsubsection*{Question 41}
\textbf{What are Throughput and Turnaround-time?} \\
\textbf{Answer:}\\ 
\textbf{Throughput} $\rightarrow$ Number of jobs done per time interval (More is better) \\
\textbf{Turnaround-time} $\rightarrow$ Difference between end and start time (Less is better)

\section*{UNIT 27}

\subsubsection*{Question 1}
\textbf{What is the key characteristic of threads within the same process regarding memory sharing?} \\
\textbf{Answer:} Threads of a process run in the same \textbf{address space.} \\
\textbf{NOTE:} Each thread has own stack, which stores local variables and manages functions calls. \\
\textbf{NOTE:} All threads have their own \textbf{virtual CPU} \\
\textbf{NOTE:} All threads are accessing the \textbf{same heap}. \\
\textbf{NOTE:} Process and thread(lower overhead) life-cycle are similar.

\subsubsection*{Question 2}
\textbf{When does parallel programming pay off?} \\
\textbf{Answer:} When the computer has more than one CPU core.

\subsubsection*{Question 3}
\textbf{What is Amdahl's Law in the context of parallel computing?} \\
\textbf{Answer:} A principle that defines the theoretical limit on the speedup that can be achieved by parallelizing a program \\
\textbf{NOTE:} N $\rightarrow$ number of processors or cores \\
\textbf{NOTE:} P $\rightarrow$ the portion of the program that can be parallelized \\
\textbf{NOTE:} 1-P $\rightarrow$ the part of the code that cannot be
parallelized.

\subsubsection*{Question 4}
\textbf{What is data decomposition?} \\
\textbf{Answer:} Breaking down large datasets into smaller parts that threads can process independently by different threads or processors. \\
\textbf{NOTE:} FORMS: cells, rows, columns, blocks

\subsubsection*{Question 5}
\textbf{What is Wall-Clock Time} \\
\textbf{Answer:} The actual elapsed time it takes for a program to run from start to finish, including all overhead. \\
\textbf{NOTE:} includes waiting time for IO operations. \\
\textbf{NOTE:} includes CPU executions. \\
\textbf{NOTE:} Wall-clock time tells us the overall efficiency

\subsubsection*{Question 6}
\textbf{What is CPU Time} \\
\textbf{Answer:} The time used by the CPU \\
\textbf{NOTE:} The CPU time does not include waiting time for IO \\
\textbf{NOTE:} \textbf{User Time} represents the time spent executing your code \\
\textbf{NOTE:} \textbf{System Time} represents the time the operating system spends handling tasks for your program. \\
\textbf{NOTE:} CPU Time helps identify bottlenecks in the processing logic.

\subsubsection*{Question 7}
\textbf{What the command time do?} \\
\textbf{Answer:} The command returns the overall execution time. \\
\textbf{NOTE:} \\
\textbf{User:} time spent in user mode \\
\textbf{Sys:} time spent in kernel mode \\
\textbf{Real:} the wall-clock time \\

\subsubsection*{Question 8}
\textbf{What is the purpose of using the clock\_gettime?} \\
\textbf{Answer:} To measure the elapsed time of a specific part of the program.

\subsubsection*{Question 9}
\textbf{What does the CLOCK\_MONOTONIC parameter in clock\_gettime ensure?} \\
\textbf{Answer:} That the measured time is independent of system clock changes.

\subsubsection*{Question 10}
\textbf{What are the advantages of using the median instead of the mean to analyze multiple measurements?} \\
\textbf{Answer:} The median is less sensitive to outliers compared to the mean. \\
\textbf{NOTE:} mean $\rightarrow$ average time

\section*{UNIT 30}

\subsubsection*{Question 1}
\textbf{How is working memory typically organized from a hardware perspective?} \\
\textbf{Answer:} As linear address space \\
\textbf{NOTE:} Memory with \textbf{2hochn} cells requires \textbf{n-Bit} address width.

\subsubsection*{Question 2}
\textbf{What is the primary goal of memory management in an operating system?} \\
\textbf{Answer:} To efficiently manage the allocation and deallocation of memory to running processes.

\subsubsection*{Question 3}
\textbf{What are the key characteristics of ideal working memory?} \\
\textbf{Answer:} \\
$\rightarrow$ Limitless size \\
$\rightarrow$ Infinity fast \\
$\rightarrow$ Free of cost \\
$\rightarrow$ Persistent \\
$\rightarrow$ Protected

\subsubsection*{Question 4}
\textbf{Which level of the memory hierarchy offers the fastest access speed?} \\
\textbf{Answer:} Registers \\
\textbf{NOTE:} Magnetic tapes (Largest storage capacity)

\subsubsection*{Question 5}
\textbf{Which component of the memory hierarchy is directly managed by the operating system?} \\
\textbf{Answer:} Main Memory and lower layers

\subsubsection*{Question 6}
\textbf{How does the absence of memory abstraction impact program execution?} \\
\textbf{Answer:} \\ 
Only one program can be executed at a time. \\
Programs will have direct access to the main memory.

\subsubsection*{Question 7}
\textbf{How does static relocation address solve the relocation problem?} \\
\textbf{Answer:} By adding a constant value to all memory addresses within a program at load time.

\subsubsection*{Question 8}
\textbf{What is the purpose of memory abstraction in operating systems?} \\
\textbf{Answer:} To provide each process with its own logical address space.

\subsubsection*{Question 9}
\textbf{What is the role of the Memory Management Unit (MMU)?} \\
\textbf{Answer:} \\
To translate logical addresses generated by a program into physical addresses. \\
Uses page tables for address mapping.

\subsubsection*{Question 10}
\textbf{What are "base" and "limit" registers used for in memory management?} \\
\textbf{Answer:} To define the starting address and size of a process's relative memory space.

\subsubsection*{Question 11}
\textbf{What is the purpose of the malloc function in C?} \\
\textbf{Answer:} To dynamically allocate a block of memory from the heap. \\
\textbf{NOTE:} Returns a pointer \\
\textbf{NOTE:} IF it fails returns zero \\
\textbf{NOTE:} Release memory $\rightarrow$ free()

\section*{UNIT 32}

\subsubsection*{Question 1}
\textbf{What is the fundamental concept of static partitioning in memory management?} \\
\textbf{Answer:} Once divided, the partitions cannot be changed, making them static \\
\textbf{NOTE:} Every partitions could hold one program \\
\textbf{NOTE:} Partions could be divided equally or unequally

\subsubsection*{Question 2}
\textbf{A process may be loaded into a partition of equal or greater size in?} \\
\textbf{Answer:} Static partition

\subsubsection*{Question 3}
\textbf{Having small amount of internal fragmentations is the weakness of} \\
\textbf{Answer:} Static partition

\subsubsection*{Question 4}
\textbf{What is the concept of dynamic partition?} \\
\textbf{Answer:} Size and number/amount of partitions are variable. \\
\textbf{NOTE:} Processes will get the amount of memory needed \\
\textbf{NOTE:} Fragmentation occurs when the process releases memory

\subsubsection*{Question 5}
\textbf{What is "fragmentation" in the context of dynamic memory allocation?} \\
\textbf{Answer:} The creation of small, unusable memory blocks between allocated partitions.

\subsubsection*{Question 6}
\textbf{What is internal fragmentation in the context of memory management?} \\
\textbf{Answer:} The wasted memory within a partition that is allocated to a process but not used by that process. \\
\textbf{NOTE:} Internal fragmentation occurs with static partitioning

\subsubsection*{Question 7}
\textbf{What is external fragmentation in memory management?} \\
\textbf{Answer:} Dynamic partitioning and refers to gaps that arise between two occupied partitions.

\subsubsection*{Question 8}
\textbf{How does a bitmap help in managing memory allocation in this approach?} \\
\textbf{Answer:} It efficiently tracks the availability of free partitions, with each bit representing the status (free(1) or occupied(0)) of a partition.

\subsubsection*{Question 9}
\textbf{How does a linked list represent memory allocation in dynamic partitioning?} \\
\textbf{Answer:} Each element of the list represents a block in working memory. \\
\textbf{NOTE:} Each list element must include a start address and the number of partitions \\
\textbf{NOTE:} The list is usually sorted according to the start addresses

\subsubsection*{Question 10}
\textbf{Double Linked lists are better that single list in case of?} \\
\textbf{Answer:} Deleting partitions

\subsubsection*{Questions 11} 
\textbf{What difficulties occurs when using bitmaps to manage partitions?} \\
\textbf{Answer:} Finding sequence of n zeros

\subsubsection*{Segmentation Strategies}
\textbf{Best Fit} $\rightarrow$ Take the smallest gap. (needs to looks all gaps) \\
\textbf{First Fit} $\rightarrow$ Take the first fitting gap. \\
\textbf{Last Fit} $\rightarrow$ Take the last fitting gap. (needs to looks all gaps) \\
\textbf{Rotating First Fit} $\rightarrow$ Take the next free gap.

\subsubsection*{Question 12}
\textbf{What is the best segmentation strategy?} \\
\textbf{Answer:} They are no the best segmentation strategy.

\subsubsection*{Question 13}
\textbf{Which statement about memory allocation is true?} \\
\textbf{Answer:} Only if all allocation request are known the ideal segmentation strategy can be chosen.

\section*{UNIT 34}

\subsubsection*{Question 1}
\textbf{What are the four main segments typically found in a process's memory space?} \\
\textbf{Answer:} Text Segment(source code), Data Segment(constants or other variables), Heap, and Stack

\subsubsection*{Question 2}
\textbf{What is the key concept behind dividing a process into pages in virtual memory?} \\
\textbf{Answer:} To allow a process to have a larger virtual address space than the available physical memory. \\
\textbf{NOTE:} Every process is divided into pages (do not need to be located sequentially in working memory).

\subsubsection*{Question 3}
\textbf{How does virtual memory overcome the limitations of physical memory size?} \\
\textbf{Answer:} By creating the illusion of a larger memory space than is physically available. \\
\textbf{NOTE:} While waiting for pages, CPU can work on another process \\
\textbf{NOTE:} The address space of a process or program is divided into pages \\
\textbf{NOTE:} Each page has both a virtual address and a physical address \\
\textbf{NOTE:} Pages represent complete virtual memory. \\
\textbf{NOTE:} Each page has a starting address and an ending address \\
\textbf{NOTE:} Not every page has to be in the main memory \\
\textbf{NOTE:} If a page is not in the physical working memory but exists only in the virtual address space, it can be swapped to the hard disk \\
\textbf{NOTE:} Accessing a swapped page on the hard drive directly is impossible. \\
\textbf{NOTE:} OS loads missing pages from hard drive to memory

\subsubsection*{Question 4}
\textbf{What is the role of the MMU (Memory Management Unit)} \\
\textbf{Answer:} \\
Translates logical to physical addresses. \\
Uses page tables for address mapping.

\subsubsection*{Question 5}
\textbf{What is a "page fault" in the context of virtual memory?} \\
\textbf{Answer:} An event that occurs when a process attempts to access a page that is not currently loaded into physical memory.

\subsubsection*{Question 6}
\textbf{Because of virtual memory the memory can be shared among?} \\
\textbf{Answer:} processes \\
\textbf{NOTE:} virtual addresses $\rightarrow$ always incoming \\
\textbf{NOTE:} physical addresses $\rightarrow$ always outgoing \\
\textbf{NOTE:} virtual addresses are one bit longer

\subsubsection*{Question 7}
\textbf{The size of page is typically?} \\
\textbf{Answer:} Power of 2

\subsubsection*{Question 8}
\textbf{What are the elements/ What is the process in Page Table Entry?} \\
\textbf{Answer:} \\
\textbf{Page frame number} (Identifies the page) \\
\textbf{Present/Absent Bit} (Indicates if the page is currently loaded in the physical memory) \\
\textbf{Protection Bit} (Defines the access permission for this page) \\
\textbf{Referenced Bit (R-Bit)} (Indicates whether the page has been read/accessed recently. \\
\textbf{Modified Bit (M-Bit)} (Shows whether the page has been modified) \\
\textbf{Caching disabled} (indicated whether caching is allowed for this page)

\subsubsection*{Question 9}
\textbf{What the paging can create?} \\
\textbf{Answer:} A bottleneck \\
\textbf{NOTE:} Converting virtual and physical addresses should be as fast as possible.
\textbf{NOTE:} All commands are located in the main memory \\
\textbf{NOTE:} Many operands are located in the main memory \\
\textbf{NOTE:} Always accessing page table is very expensive

\subsubsection*{Question 10}
\textbf{What is the primary function of the Translation Lookaside Buffer (TLB)?} \\
\textbf{Answer:} To speed up virtual-to-physical address translation by caching recently accessed page table entries. \\
\textbf{NOTE:} Part of the MMU with 64 entires of last used pages. \\
\textbf{NOTE:} All entries are compared in parallel \\
\textbf{NOTE:} Accessing page table is not required if it is found in TLB, otherwise it will be replaced \\
\textbf{NOTE:} When deleting an entry in TLB, M-BIT of the page is set.

\section*{UNIT 37}

\subsubsection*{NOTE}
The kernel creates a new entry in the process table and immediately allocates part of the processor power to this new process. This \textbf{kernel task} is called \textbf{scheduling} and is already supported by the hardware on modern processors. \\
\textbf{Operating System} is responsible for \textbf{scheduling}. 
\textbf{Main Tasks of Operating System:} \\
$\rightarrow$ \textbf{Abstraction} \\
$\rightarrow$ \textbf{Virtualization} \\
$\rightarrow$ \textbf{Source management} \\ \\
\textbf{PID} $\rightarrow$ processID \\
\textbf{TTY} $\rightarrow$ shows the console \\
\textbf{PTS} $\rightarrow$ pseudo terminal \\
\textbf{VSZ} $\rightarrow$ virtual memory (in kilobytes) \\
\textbf{RSS} $\rightarrow$ real memory (in kilobytes) \\ \\
\textbf{Stat = Status} \\
\textbf{R} = Running or about to run \\
\textbf{D/I} Uninterrupted sleep (wait for I/O) \\
\textbf{S} = Interrupted sleep (wait for event) \\
\textbf{T} = Stopped \\ \\
\textbf{/number} $\rightarrow$ running on different core \\ \\
\textbf{Process Priority and Niceness:} \\
\textbf{Highest priority:} 0 \\
\textbf{Lowest Priority:} 39 \\
A process with a high priority is executed before a low priority. \\
\textbf{Processes} with the \textbf{same priority} are executed in \textbf{sequence} \\
\textbf{Kernel} can also \textbf{change the priority}, not the “nice” value \\
\textbf{Priority = Niceness + 20} \\
Command \textbf{renice} to modify the nice value \\ \\
\textbf{SIGNALS:} \\
\textbf{SIGINT(2)} $\rightarrow$ Interrupt from keyboard (CTRL-C) \\
\textbf{SIGQUIT(3)} $\rightarrow$ Quit from keyboard (CTRL-slash) \\
\textbf{SIGKILL(9)} $\rightarrow$ Kill, unblockable (terminates direct the process) \\
\textbf{SIGTERM(15)} $\rightarrow$ Termination request \\
\textbf{SIGTSTP(20)} $\rightarrow$ Stop from keyboard (CTRL-Z) (go to state T) \\ \\
\textbf{SIGNALS IN C PROGRAMS} \\
Parameter \textbf{sig} contains the \textbf{signal number} \\
Parameter \textbf{func} contains a \textbf{pointer to function} \\
Function \textbf{signal} returns a \textbf{pointer to the last handler} \\ \\
\textbf{SIG\_DFL} $\rightarrow$ default action handler. \\
\textbf{SIG\_IGN} $\rightarrow$ ignore the signal. \\ \\
\textbf{FILES in proc Directory} \\
\textbf{cpuinfo} $\rightarrow$ info about processor or cores \\
\textbf{devices} $\rightarrow$ available devices (hard-disk) \\
\textbf{cmdline} $\rightarrow$ parameters, which kernel was started \\
\textbf{meminfo} $\rightarrow$ memory management \\
\textbf{version} $\rightarrow$ version of Linux kernel \\ \\
\textbf{Changing Kernel Configuration} \\
Kenel configuration available in \textbf{/proc/sys} \\
Many files contain switches \textbf{(0 = false, 1 = true)}


\subsubsection*{Question 1}
\textbf{What is the primary role of the kernel in process management?} \\
\textbf{Answer:} To create and manage processes, allocate resources, and schedule processor time.

\subsubsection*{Question 2}
\textbf{What is the primary function of the ps command in Linux?} \\
\textbf{Answer:} To display information about currently running processes.

\subsubsection*{Question 3}
\textbf{What does the "PID" column in the output of the ps command represent?} \\
\textbf{Answer:} The unique process identifier (processID). 

\subsubsection*{Question 4}
\textbf{What does the command ps -u $<$username$>$ do?} \\
\textbf{Answer:} Shows all processes that are own by this user.

\subsubsection*{Question 5}
\textbf{What does the command ps au do?} \\
\textbf{Answer:} Shows all the processes, not only those that are connected to any kind of console, but all processes even if they are running in the background.

\subsubsection*{Question 6}
\textbf{What does the command ps aux do?} \\
\textbf{Answer:} Shows all real processes (currently running on the system.

\subsubsection*{Question 7}
\textbf{What does the "R" state indicate in the process status column of the ps command output?} \\
\textbf{Answer:} The process is currently running or is ready to run (runnable).

\subsubsection*{Question 8}

\textbf{What does the "D/I" state indicate in the process status column of the ps command output?} \\
\textbf{Answer:} The process is in uninterruptible sleep, typically waiting for I/O operations to complete.

\subsubsection*{Question 9}

\textbf{What does the "S" state indicate in the process status column of the ps command output?} \\
\textbf{Answer:} The process is in interruptible sleep, waiting for an event or signal.

\subsubsection*{Question 10}
\textbf{What does the "T" state indicate in the process status column of the ps command output?} \\
\textbf{Answer:} The process has been stopped or suspended.

\subsubsection*{Question 11}
\textbf{What is the primary function of the top command in Linux?} \\
\textbf{Answer:} To display real-time information about system performance and running processes.

\subsubsection*{Question 12}
\textbf{What does the "load average" displayed by top represent?} \\
\textbf{Answer:} \\
The first unit $\rightarrow$ load in the last minute \\
The second unit $\rightarrow$ load in the last 5 minute \\
The third unit $\rightarrow$ load in the last 15 minute \\
\textbf{NOTE:} If one CPU is fully occupied by the processes, then the value would be 1.0

\subsubsection*{Question 13}
\textbf{What indicates Zombie processes?} \\
\textbf{Answer} Zombie processes are those that are killed because of an error

\subsubsection*{Question 14}
\textbf{What does the command yes do?} \\
\textbf{Answer:} Overload

\subsubsection*{Question 15}
\textbf{What command can be used to send a process to the background and foreground?} \\
\textbf{Answer:} bg (background) fg (foreground)

\subsubsection*{Question 16}
\textbf{What is the relationship between "nice" value and process priority in Linux?} \\
\textbf{Answer:} Priority = Nice Value + 20

\subsubsection*{Question 17}
\textbf{How can you change the "nice" value of a process in Linux?} \\
\textbf{Answer:} Using the renice command.

\subsubsection*{Question 18}
\textbf{How does the "nice" value affect the scheduling of a process?} \\
\textbf{Answer:} A higher "nice" value generally reduces the process's priority, leading to less CPU time allocation.

\subsubsection*{Question 19}
\textbf{What is a signal in the context of operating systems?} \\
\textbf{Answer:} An asynchronous notification sent to a process to indicate a specific event or condition.

\subsubsection*{Question 20}
\textbf{What is the "proc" file system in Linux?} \\
\textbf{Answer:} A virtual file system that provides access to kernel data structures and information about running processes. \\
\textbf{NOTE:} Contains information about all processes. \\
\textbf{NOTE:} Kernel publishes information about the system.

\subsubsection*{Question 21}
\textbf{How does the "proc" file system differ from a traditional file system?} \\
\textbf{Answer:} It does not reside on physical storage; its contents are dynamically generated by the kernel.

\subsubsection*{Question 22}
\textbf{What information provides file /proc/PID/cmdline?} \\
\textbf{Answer:} Program name and command line parameters.

\subsubsection*{Question 23}
\textbf{What contains file /proc/sys/kernel/panic} \\
\textbf{Answer:} number of seconds before reboot.

\subsubsection*{Question 24}
\textbf{What is the purpose of the /proc/sys directory in the Linux file system?} \\
\textbf{Answer:} To provide access to and modify kernel parameters.

\subsubsection*{Question 25}
\textbf{What is the significance of the /proc/sys/kernel/ctrl-alt-del file?} \\
\textbf{Answer:} It defines the system's behavior when the CTRL+ALT+DEL key combination is pressed.

\section*{QUIZ Questions}

\subsubsection*{Question 1}
\textbf{What are some advantages of parallel programming?} \\
\textbf{Answer:} \\ 
Better handling of large datasets. \\
Improved performance with multiple cores.

\subsubsection*{Question 2}
\textbf{What is the role of base and limit registers?} \\
\textbf{Answer:} Define the start and size of a process memory space.

\subsubsection*{Question 3}
\textbf{What is a key advantage of TLB?} \\
\textbf{Answer:} Reduces the frequency of page table access.

\subsubsection*{Question 4}
\textbf{What is the purpose of Translation Look aside Buffer?} \\
\textbf{Answer:} \\ 
Caches recently used page table entries. \\
Speeds up virtual-to-physical address translation.

\subsubsection*{Question 5}
\textbf{What is the concept of a process in an operating system?} \\
\textbf{Answer:} A program that is currently being executed.

\subsubsection*{Question 6}
\textbf{Which statements about paging are true} \\
\textbf{Answer:} \\
Paging abstracts physical memory into virtual memory. \\
It leads to potential challenges like page faults.

\subsubsection*{Question 7}
\textbf{What is the role of virtual memory?} \\
\textbf{Answer:} \\
Allows processes to use isolated address spaces. \\
Extends the physical memory available to processes.

\subsubsection*{Question 8}
\textbf{Which file in the proc directory contains information about the CPU?} \\
\textbf{Answer:} cpuinfo

\subsubsection*{Question 9}
\textbf{What are the components of CPU time?} \\
\textbf{Answer:} \\
User time \\
System time

\subsubsection*{Question 10}
\textbf{What is the function of pages in virtual memory?} \\
\textbf{Answer:} \\
Represent a complete segment of virtual memory. \\
Divide a process's space into blocks.

\subsubsection*{Question 11}
\textbf{What challenges arise with frequent scheduling?} \\
\textbf{Answer:} \\
Trade-off between responsiveness and efficiency. \\
Increasing overhead in storing/restoring states. \\
Current CPU cache becomes invalid.

\subsubsection*{Question 12}
\textbf{What happens when the quantum is too small?} \\
\textbf{Answer:} \\
Processes are switched often. \\
Response times are shorter.

\subsubsection*{Question 13}
\textbf{What is the main advantage of Lottery Scheduling?} \\
\textbf{Answer:} \\
Provides probabilistic fairness among processes. \\
Allowing dynamic adjustment of process priorities.

\subsubsection*{Question 14}
\textbf{What defines the state of a process?} \\
\textbf{Answer:} Process Control Block (PCB)

\subsubsection*{Question 15}
\textbf{What is the relationship between priority and niceness in Linux?} \\
\textbf{Answer:} \\
Priority \\
Niceness + 20 \\
Niceness - 20 \\
Niceness * 20

\subsubsection*{Question 16}
\textbf{How does a pre-emptive scheduling works?} \\
\textbf{Answer:} \\
A timer interrupts triggers scheduling. \\
A process is interrupted by the OS.

\subsubsection*{Question 17}
\textbf{What does the proc directory represent in Linux?} \\
\textbf{Answer:} A virtual file system for system information.

\subsubsection*{Question 18}
\textbf{What happens in systems without memory abstraction?} \\
\textbf{Answer:} \\ 
Only one program is executed at a time. \\
Programs have direct access to main memory.

\subsubsection*{Question 19}
\textbf{How does Round-Robin scheduling handle CPU-bound and IO-bound processes?} \\
\textbf{Answer:} \\
IO-bound processes often finish their quantum early. \\
CPU-bound may have higher response times. 

\subsubsection*{Question 20}
\textbf{Which statements are true about Lottery Scheduling?} \\
\textbf{Answer:} \\
Processes are assigned tickets representing their chance of execution. \\
Process with more tickets has higher chance of being selected.

\subsubsection*{Question 21}
\textbf{Which of these statements about swapping are correct?} \\
\textbf{Answer:} \\
Swapping moves processes between memory and disk. \\
Swapping allows multiple programs to run on limited memory.

\subsubsection*{Question 22}
\textbf{What happens if a process does not finish within its time quantum in Round-Robin scheduling?} \\
\textbf{Answer:} It is moved to the end of the ready queue.

\subsubsection*{Question 23}
\textbf{What does Amdahl's low about speed-up?} \\
\textbf{Answer:} \\ 
Adding processes does not always scale linearly. \\
It is limited by the non-parallel portion.

\subsubsection*{Question 24} 
\textbf{What should be done for reliable time measurements in noisy environment?} \\
\textbf{Answer:} \\
Repeat measurements 100 to 1000 times. \\
Calculate median or mean values.

\subsubsection*{Question 25}
\textbf{What is the role of the Memory Management Unit(MMU)} \\
\textbf{Answer:} \\ 
Translates logical to physical addresses. \\
Uses page tables for address mapping.

\subsubsection*{Question 26}
\textbf{What are transitions between process states cause by?} \\
\textbf{Answer:} \\
Timer interrupts. \\
Incoming data. \\
I/O system calls.

\subsubsection*{Question 27}
\textbf{Which command is used to visualize system load?} \\
\textbf{Answer:} top/htop.

\subsubsection*{Question 28}
\textbf{What happens when User + Sys $<$ Real?} \\
\textbf{Answer:} Program is waiting for resources.

\subsubsection*{Question 29}
\textbf{Which of the following are challenges in memory management?} \\
\textbf{Answer:} \\
Allocating and releasing memory. \\
Avoiding memory leaks.

\subsubsection*{Question 30}
\textbf{Which of these are characteristics of ideal memory?} \\
\textbf{Answer:} \\
Infinity size. \\
Infinity fast access. \\
Persistence after power outages.

\subsubsection*{Question 31}
\textbf{Which statements best describes static relocations?} \\
\textbf{Answer:} Adjust addresses during program loading.

\subsubsection*{Question 32}
\textbf{What is the key characteristics of Round-Robin scheduling?} \\
\textbf{Answer:} \\
Processes are assigned fixed time slices (quantum). \\
Time slices fairness among processes.

\subsubsection*{Question 33}
\textbf{What is the purpose of the yes command?} \\
\textbf{Answer:} to create system load for testing.

\subsubsection*{Question 34}
\textbf{Which of the following statements about memory is true?} \\
\textbf{Answer:} Memory is logically organized as a linear address space.

\subsubsection*{Question 35}
\textbf{What is the potential drawback of Priority Scheduling?} \\
\textbf{Answer:} Processes with lower priority may starve.

\subsubsection*{Question 36}
\textbf{Which components are part of the process model?} \\
\textbf{Answer:} \\
Program counter. \\
Content of registers. \\
Virtual CPU.

\subsubsection*{Question 37}
\textbf{What are examples of data decomposition?} \\
\textbf{Answer:} \\ Rows, Columns, Blocks

\subsubsection*{Question 38}
\textbf{Which command displays detailed information about processes?} \\
\textbf{Answer:} ps, top

\subsubsection*{Question 39}
\textbf{What happens when a page fault occurs?} \\
\textbf{Answer:} \\
The OS loads the required page from the disk. \\
The CPU pauses execution of the process.

\subsubsection*{Question 40}
\textbf{How do Lottery Scheduling and Priority Scheduling differ in terms of fairness?} \\
\textbf{Answer:} \\
Lottery Scheduling provides probabilistic fairness. \\
Priority Scheduling may cause starvation for low-priority processes.

\subsubsection*{Question 41}
\textbf{What does SIGKILL signal do?} \\
\textbf{Answer:} Terminates a process unconditionally.

\subsubsection*{Question 42}
\textbf{Which of the following statements about page tables are true?} \\
\textbf{Answer:} \\ 
They map virtual addresses to physical addresses. \\
They are stored in the main memory.

\subsubsection*{Question 43}
\textbf{Which signal is sent by CTRL-C} \\
\textbf{Answer:} SIGINT

\subsubsection*{Question 44}
\textbf{Which scheduling strategies are used for batch processing?} \\
\textbf{Answer:} \\
First-Come, First-Served (FCFS) \\
Shortest Processing Time (SPT) \\
Shortest Job First (SJF)

\subsubsection*{Question 45}
\textbf{What must be done after using malloc?} \\
\textbf{Answer:} Use free to release memory.

\subsubsection*{Question 46}
\textbf{What does the function malloc do in C?} \\
\textbf{Answer:} \\
Allocates memory dynamically. \\
Returns a pointer to allocate memory.

\subsubsection*{Question 47}
\textbf{Which formula represents Amdahl's low?} \\
\textbf{Answer:} S = 1 / ((1 - P) + (P / N))

\subsubsection*{Question 48}
\textbf{What is the effect of setting /proc/sys/kernel/ctrl-alt-del to 1?} \\
\textbf{Answer:} Enables immediate system reboot via CTRL-ALT-DEL

\subsubsection*{Question 49}
\textbf{What is the impact of using CLOCK\_MONOTONIC in time measurements?} \\
\textbf{Answer:} Avoids system time changes.

\subsubsection*{Question 50}
\textbf{What does the Modified Bit (M-Bit) in a page table indicates?} \\
\textbf{Answer:} Whether the page has been written to.

\subsubsection*{Question 51}
\textbf{What is a relocation problem in memory managements?} \\
\textbf{Answer:} Adjusting memory addresses when processes are moved.

\subsubsection*{Question 52}
\textbf{Which of the following scenarios might lead to starvation in Priority Scheduling?} \\
\textbf{Answer:} Low-priority processes are always pre-emptive by higher-priority ones.

\subsubsection*{Question 53}
\textbf{Which process status indicates a process is waiting for IO?} \\
\textbf{Answer:} S Interruptible sleep.

\subsubsection*{Question 54}
\textbf{Which statements can mitigate starvation in Priority Scheduling?} \\
\textbf{Answer:} \\
Use aging to increase priority of long-waiting processes. \\
Remove all priorities and use Round-Robin instead.

\subsubsection*{Question 55}
\textbf{Why is a one-time measurement insufficient?} \\
\textbf{Answer:} \\
Variability in execution time. \\
External noise can affect results.

\subsubsection*{Question 56}
\textbf{What is the purpose of a scheduler?} \\
\textbf{Answer:} \\ 
Selects the next process execution. \\
Manages the process queue.

\subsubsection*{Question 57}
\textbf{What does the cwd file in a process's proc directory indicates?} \\
\textbf{Answer:} The current working directory of the process.

\subsubsection*{Question 58}
\textbf{What are examples of CPU-bound processes?} \\
\textbf{Answer:} \\
Processes simulating physics. \\
Processes performing intensive calculations.

\subsubsection*{Question 59}
\textbf{What does a Present/Absent Bit in a page table indicates?} \\
\textbf{Answer:} Whether the page is in main memory.

\subsubsection*{Question 60}
\textbf{Which practices help prevent memory leaks in C?} \\
\textbf{Answer:} \\
Write free immediately after malloc. \\
Use consistent memory allocation patterns.

\subsubsection*{Question 61}
\textbf{Which of the following describes kernel mode?} \\
\textbf{Answer:} \\
Manages scheduling and memory. \\
Runs in privileges mode.

\subsubsection*{Question 62}
\textbf{What are the challenges of large page tables?} \\
\textbf{Answer:} \\
They consume significant memory space. \\
Accessing them repeatedly slows performance.

\subsubsection*{Question 63}
\textbf{What is the role of the dispatcher in process scheduling?} \\
\textbf{Answer:} \\ 
Gives control to the next process. \\
Saves the state of the current process. \\
Loads the last state of the next process.

\subsubsection*{Question 64}
\textbf{What is the main advantage of threads sharing the same address space?} \\
\textbf{Answer:} Easier data sharing between threads.

\subsubsection*{Question 65}
\textbf{Which files in the proc directory provide process-specific information?} \\
\textbf{Answer:} status





\end{document}