\documentclass[a4paper,12pt]{article}
\usepackage[utf8]{inputenc}
\usepackage[english]{babel}
\usepackage{geometry}
\geometry{margin=1in}
\usepackage{lmodern}
\usepackage{parskip}
\usepackage{setspace}
\onehalfspacing

\title{Technical Presentation \\
	Abstract}
\author{
	Group Members: \\
	Enrico Ebert, enrico.ebert@study.thws.de, 5123098 \\
	Kristian Popov, kristian.popov@study.thws.de, 5123029 \\
	Glison Doci, glison.doci@study.thws.de, 5123136 \\
	Orik Mazreku, orik.mazreku@study.thws.de, 5123144
}
\date{\today}

\begin{document}
	
	\maketitle

\section*{Abstract}	
AI coding assistants use artificial intelligence to help developers with a variety of coding tasks. They can analyze code, suggest improvements, spot bugs, offer fixes, or even generate entire code snippets based on just a prompt. You’ll often find them built right into integrated development environments (IDEs) or available as standalone tools, supporting a wide range of programming languages. Some popular examples include GitHub Copilot, Tabnine, and Microsoft IntelliCode. \\
However, integrating AI into development workflows also presents challenges. High upfront costs, technical complexity, and concerns about data privacy and security must be carefully considered. Despite these drawbacks, mastering AI tools provides a significant competitive advantage, enabling developers to build more advanced, user-friendly, and innovative websites. This presentation explores how AI assistants are reshaping frontend development, highlighting both their potential and their limitations.

\end{document}